% 400 - 600 palabras

% NOTAS:
%   Describir el contexto y los antecedentes 
%   Por que es interesante realizar este estudio
%   Introducir al tema de manera no necesariamente técnica
%   Desde conceptos generales a particulares
%   No es necesario ahondar en conceptos generales

%   Incluye antecedentes asociados al proyecto:
%       - Motivación
%       - Significancia 
%       - Originalidad
%       - Propósito y los objetivos del desarrollo.
%       - Lo que se propone realizar, delimitando:
%           - Alcance 

%% PREGUNTAS GUÍAS
%   - ¿Cuál es el tema y el entorno del proyecto?
%   - ¿Para qué sirve el desarrollo que se plantea?
%   - ¿Cémo se pretende resolver de forma general el proyecto?
%   - ¿Cuál es el método empleado?
%   - ¿Cuáles son las limitaciones del proyecto?


\section{Introducción}
Dentro del archivo .tex podrás encontrar unas preguntas guías sobre como redactar esta sección.

Ejemplos de infinitivo
\begin{itemize}
    \item Mal: El sistema deberá irrigar agua
    \item Mal: El sistema irrigará agua
    \item Mal: El sistema se diseñó pensando en que irrigará agua
    \item Bien: El sistema está diseñado para irrigar agua 
    \item Bien: El sistema irriga agua
\end{itemize}