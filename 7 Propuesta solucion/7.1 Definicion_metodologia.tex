
% Aquí se elige la metodología mecatrónica
\subsection{Definición de la metodología mecatrónica}

Durante el desarrollo del proyecto se usa la metodología VDI-2206 mostrada en la Figura \ref{VDI2206}. La cual consta de las siguientes etapas:




\begin{itemize}
    %\item Requerimientos - El objeto es descrito de forma precisa en forma de requerimientos, los cuales al mismo tiempo forman el parámetro con el que el producto será juzgado.
    \item Diseño del sistema - Se establece un concepto solución  que describe las características operacionales principales del sistema físico y lógico.
    \item Diseño del dominio específico - Se desarrollan interpretaciones y cálculos más detallados para asegurar el cumplimiento de la función principal.
    \item Integración del sistema - Los resultados individuales se unen para formar un sistema sinérgico.
    \item Validación y Verificación - El progreso realizado con el diseño debe ser continuamente verificado respecto a los requerimientos del concepto solución.
    \item Modelado y análisis del modelo - Las fases descritas son abordadas mediante modelos y herramientas computacionales.
    %\item Producto - Es el resultado final.
\end{itemize}

Las cuales tienen como entrada los Requerimientos y como salida el Producto. Los requerimientos describen el objeto de forma precisa, los cuales al mismo tiempo forman el parámetro con el que el producto será juzgado y el producto es el resultado final de las etapas elaboradas.
\begin{figure}[htbp]
    \includegraphics[scale=0.45]{7 Propuesta solucion/7_Imagenes/VDI 2206 Español Im.JPG}
    \caption{Modelo VDI-2206.}
    \label{VDI2206}
\end{figure}